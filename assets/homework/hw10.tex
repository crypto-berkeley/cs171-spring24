\documentclass[11pt]{article}

\usepackage{amsmath}
\usepackage{amssymb}
\usepackage{enumerate,comment}
\usepackage{url}
\usepackage{color}
\usepackage[margin=1.2in]{geometry}
\usepackage{hyperref}

\usepackage{fancyhdr}
\pagestyle{fancy}
\setlength{\headheight}{20pt}
\setlength{\headsep}{20pt}
\fancyhead[L]{\small{CS 171, Spring 2024}}
\fancyhead[R]{\small{Prof. Sanjam Garg}}

\newcommand{\Gen}{\mathsf{Gen}}
\newcommand{\Enc}{\mathsf{Enc}}
\newcommand{\enc}{\mathsf{Enc}}
\newcommand{\Dec}{\mathsf{Dec}}
\newcommand{\dec}{\mathsf{Dec}}
\newcommand{\Setup}{\mathsf{Setup}}
\newcommand{\Commit}{\mathsf{Commit}}
\newcommand{\Open}{\mathsf{Open}}
\newcommand{\Z}{\mathbb{Z}}
\newcommand{\cM}{\mathcal{M}}
\newcommand{\cC}{\mathcal{C}}
\newcommand{\cK}{\mathcal{K}}
\newcommand{\cL}{\mathcal{L}}
\newcommand{\Rank}{\mathsf{Rank}}
\newcommand{\F}{\mathbb{F}}
\newcommand{\getsr}{\stackrel{\$}{\gets}}
\newcommand{\A}{\mathcal{A}}
\newcommand{\D}{\mathcal{D}}
\newcommand{\G}{\mathcal{G}}
\newcommand{\GG}{\mathbb{G}}
\renewcommand{\H}{\mathcal{H}}
\newcommand{\GF}{\mathsf{GF}}
\newcommand{\msg}{\mathsf{msg}}
\newcommand{\st}{\mathsf{st}}
\newcommand{\pk}{\mathsf{pk}}
\newcommand{\srs}{\mathsf{srs}}
\newcommand{\pp}{\mathsf{pp}}
\newcommand{\sk}{\mathsf{sk}}
\newcommand{\ct}{\mathsf{ct}}
\newcommand{\ZZ}{\mathbb{Z}}

\newcommand{\sfa}{\mathsf{a}}
\newcommand{\sfb}{\mathsf{b}}
\newcommand{\sfc}{\mathsf{c}}
\newcommand{\sft}{\mathsf{t}}
\newcommand{\sfx}{\mathsf{x}}
\newcommand{\sfy}{\mathsf{y}}
\newcommand{\sfz}{\mathsf{z}}

\newcommand{\secp}{\lambda}
\newcommand{\negl}{\mathsf{negl}}
\newcommand{\nonnegl}{\mathsf{non}\text{-}\mathsf{negl}}


\newcommand{\bin}{\{0,1\}}
\newcommand{\bit}{\bin}
\newcommand{\B}{\mathcal{B}}
\newcommand{\duedate}{April 25, 2024 at 8.59pm via Gradescope}

\newcommand{\gen}{\mathsf{Gen}}
\newcommand{\mac}{\mathsf{Mac}}
\newcommand{\sign}{\mathsf{Sign}}
\newcommand{\verify}{\mathsf{Vrfy}}
\newcommand{\C}{\mathcal{C}}

\newcommand{\qed}{\hspace*{\fill}\rule{7pt}{7pt}}
\newtheorem{theorem}{Theorem}[section]
\newtheorem{definition}[theorem]{Definition}
\newtheorem{corollary}[theorem]{Corollary}
\newtheorem{lemma}[theorem]{Lemma}
\newtheorem{claim}[theorem]{Claim}
\newtheorem{fact}[theorem]{Fact}
\newtheorem{conjecture}[theorem]{Conjecture}
\newtheorem{remark}[theorem]{Remark}
\newenvironment{assumption}{\noindent{\bf Assumption}\hspace*{1em}\begin{em}}{\end{em}\medskip}
\numberwithin{equation}{section}
\newenvironment{proof}{\noindent{\bf Proof}\hspace*{1em}}{\qed\medskip}
\newenvironment{solution}{\noindent{\bf Solution}\hspace*{1em}}{\qed\medskip}

\begin{document}



\section*{CS 171: Problem Set 10\\ {\small Due Date: \duedate} }

\section{Proof of Decryption (10 Points)}
We will construct a zero-knowledge proof system for DDH triples. This can be used to prove that a given El Gamal ciphertext was decrypted correctly without revealing the secret decryption key.\\

Let $\pp = (\GG, q, g) \leftarrow \mathcal{G}(1^n)$ be a group in which DDH is hard. Let $\cL$ be the language of DDH triples for this group:
\[\cL = \{(\pp, g^\sfa, g^\sfb, g^\sfc) : \sfc = \sfa \cdot \sfb \mod q\}\]

Given an instance $x = (\pp, g^\sfa, g^\sfb, g^\sfc) \in \cL$, let the corresponding witness be $w = \sfb$. The witness provides a simple way to verify that $x \in \cL$:
\[R(x, w) = 
\begin{cases}
    1 \text{ if } & g^{w} = g^\sfb \text{ and } (g^\sfa)^w = g^{\sfc}\\
    0 & \text{ else }
\end{cases}\]

We can also prove that $x \in \cL$ without revealing the witness to the verifier. To do so, we will construct a zero-knowledge proof below.\\

\noindent\underline{A Zero-Knowledge Protocol for $\cL$:}
\begin{itemize}
    \item \underline{Inputs:} The prover $P$ takes inputs $(1^\secp, x, w)$ and the verifier $V$ takes inputs $(1^\secp, x)$. $x = (\pp, g^\sfa, g^\sfb, g^\sfc)$, and $w \in \ZZ_q$.
    \item $P$ samples $\sfx \leftarrow \ZZ_q$, computes $\sft = \sfa \cdot \sfx \mod q$, and sends $(g^\sfx,g^{\sft})$ to $V$. 
    \item $V$ samples $\sfy \leftarrow \mathbb{Z}_q$ and sends $\sfy$ to $P$.
    \item $P$ computes $\sfz = w \cdot \sfy + \sfx$ and sends $\sfz$ to $V$.
    \item Final step: TBD
\end{itemize}

\paragraph{Questions:} 
\begin{enumerate}
    \item Fill in the final step to show how $V$ should verify the proof $(g^\sfx,g^{\sft}, \sfy, \sfz)$.
    \item Show that this proof system satisfies completeness and soundness.
    \item Show that this proof system satisfies honest-verifier zero-knowledge.
\end{enumerate}

The definitions of completeness, soundness, and honest-verifier zero-knowledge are given in Discussion 11.
\pagebreak


\section{Hiding and Binding For KZG Commitments (15 Points)}
In discussion 11, we showed that the basic KZG commitment protocol is not hiding because the $\mathsf{Commit}$ function is deterministic. In section \ref{sec:randomized-KZG-commitment-construction} below, we give a modified version of the scheme in which the $\mathsf{Commit}$ function is randomized.

\paragraph{Question:} Prove that the commitment scheme given in section \ref{sec:randomized-KZG-commitment-construction} satisfies the notions of hiding and polynomial binding given in section \ref{sec:hiding-and-polynomial-binding}, assuming that the $d$-discrete log problem is hard.

\subsection{A Randomized Polynomial Commitment Scheme}\label{sec:randomized-KZG-commitment-construction}

\begin{enumerate}
    \item $\Gen(1^n)$: \begin{enumerate}
        \item Let $d$ be polynomial in $n$.
        \item Set up a bilinear map by sampling 
        \[\pp = (\GG, \GG_T, q, g, e) \leftarrow \G(1^n)\]
        \item Sample $h \leftarrow \GG$ and $\tau \leftarrow \ZZ_q^*$.
        \item Finally, output
        \[\mathsf{params} = \left(\pp, g^{\tau}, g^{(\tau^2)}, \dots, g^{(\tau^{d})}, h, h^{\tau}, h^{(\tau^2)}, \dots, h^{(\tau^{d})}\right)\]
    \end{enumerate}
    \item $\Commit(\mathsf{params}, f)$:
    \begin{enumerate}
        \item Let $f$ be a polynomial $\in \ZZ_q[X]$ of degree $\leq d$:
        \[f(X) = \sum_{i = 0}^{d} \alpha_i \cdot X^i\]
        where every $\alpha_i \in \ZZ_q$.
        \item Sample a polynomial $r \in \ZZ_q[X]$ of degree $\leq d$ uniformly at random. In other words, sample $\beta_0, \dots, \beta_d \leftarrow \ZZ_q$ independently and uniformly at random, and let 
        \[r(X) = \sum_{i = 0}^d \beta_i \cdot X^i\]
        \item Compute and output the commitment:
        \begin{align*}
            \mathsf{com} &= \prod_{i = 0}^{d} \left(g^{(\tau^i)}\right)^{\beta_i} \cdot \prod_{i = 0}^{d} \left(h^{(\tau^i)}\right)^{\alpha_i}\\
            &= g^{r(\tau)} \cdot h^{f(\tau)}
        \end{align*}
    \end{enumerate}

\emph{Note}: We also define $\Commit(\mathsf{params}, f; r)$ to take the random polynomial $r$ as input, rather than sampling $r$ internally.
\end{enumerate}
\pagebreak


\subsection{Definitions}\label{sec:hiding-and-polynomial-binding}
Hiding basically says that $\Commit(f, \mathsf{params})$ doesn't reveal any information about $f$. The definition of hiding resembles the definition of CPA security.
\begin{definition}[Hiding]
$ $\\

\noindent\underline{$\mathsf{Hiding\text{-}Game}(n, \A)$:}
\begin{enumerate}
    \item The challenger samples $\mathsf{params} \leftarrow \Gen(1^n)$ and sends $\mathsf{params}$ to the adversary $\A$.
    \item $\A$ outputs two polynomials $f_0, f_1 \in \ZZ_q[X]$ of degree $\leq d$.
    \item The challenger samples $b \leftarrow \bit$ and computes:
    $\mathsf{com}^* = \Commit(\mathsf{params}, f_b)$. They send $\mathsf{com}^*$ to $\A$.
    \item $\A$ outputs a guess $b'$ for $b$. The output of the game is $1$ if $b' = b$ and $0$ otherwise.
\end{enumerate}

The commitment scheme is \textbf{hiding} if for any PPT adversary $\A$,
\[\Pr[\mathsf{Hiding\text{-}Game}(n, \A) \to 1] \leq \frac{1}{2} + \negl(n)\]
\end{definition}

Next, we'll consider a notion called polynomial binding, which says that the adversary cannot find two inputs to $\Commit$ that produce the same commitment.  This resembles the definition of collision-resistance.
\begin{definition}[Polynomial Binding]
$ $\\

\noindent\underline{$\mathsf{Polynomial\text{-}Binding\text{-}Game}(n, \A)$:}
\begin{enumerate}
    \item The challenger samples $\mathsf{params} \leftarrow \Gen(1^n)$ and sends $\mathsf{params}$ to the adversary $\A$.
    \item $\A$ outputs two pairs $(f_0, r_0)$ and $(f_1, r_1)$, where $f_0, r_0, f_1, r_1$ are polynomials $\in \ZZ_q[X]$ of degree $\leq d$.
    \item The output of the game is $1$ if $f_0 \neq f_1$, and
    \[\Commit(\mathsf{params}, f_0; r_0) = \Commit(\mathsf{params}, f_1; r_1)\]
    Otherwise, the output of the game is $0$.
\end{enumerate}
The commitment scheme satisfies \textbf{polynomial binding} if 
\[\Pr[\mathsf{Polynomial\text{-}Binding\text{-}Game}(n, \A) \to 1] \leq \negl(n)\]
\end{definition}

Finally, we will prove polynomial binding using the hardness of the following problem.
\begin{definition}[A Variant of Discrete Log]\label{def:d-discrete-log}
$ $\\
\noindent\underline{$d\text{-}\mathsf{Discrete}\text{-}\mathsf{Log}(n, \A)$:}
\begin{enumerate}
    \item Let $d$ be polynomial in $n$. 
    \item The challenger samples $\pp = (\GG, \GG_T, q, g, e) \leftarrow \G(1^n)$ as well as $\tau \leftarrow \ZZ_q$. Then they send the adversary:
    $\left(\pp, g^{\tau}, g^{(\tau^2)}, \dots, g^{(\tau^{d})}\right)$
    \item The adversary $\A$ outputs a guess $\tau'$ for $\tau$. The output of the game is $1$ if $\tau' = \tau$ and $0$ otherwise.
\end{enumerate}
The $d$-discrete-log problem is hard if for any PPT adversary $\A$, 
\[\Pr[d\text{-}\mathsf{Discrete}\text{-}\mathsf{Log}(n, \A) \to 1] \leq \negl(n)\]
\end{definition}
Note that if the $d$-discrete-log problem is hard, then in addition, the regular discrete log problem is hard for $\GG$.
\pagebreak

\section{Course Evaluation (Extra Credit: 2 Points)}
Complete your course evaluation for this course. You can write as much or as little as you want. Include a screenshot of the submission receipt when you submit this assignment to Gradescope to prove that you've finished your evaluation.\pagebreak

\end{document}